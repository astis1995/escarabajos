\subsection{Comparison with Zubov et al. claims}

\subsubsection*{Claim 1}
\textit{The new species is very close to \textit{C. resplendens} and has only few morphological differences from it. Clypeus of \textit{C. kalinini sp.n.} is slightly longer than in \textit{C. resplendens}.}

Head's vertical clipeus length, \textbf{A5}, shows a statistically significant difference. \subsubsection*{Claim 2}
\textit{Pronotum in \textit{C. kalinini sp.n.} is slightly longer in relation to its width than in \textit{C. resplendens}, its sides have smaller angles, whereas in \textit{C. resplendens} the sides of pronotum are rounded.}

None of the pronotum's vertical length---horizontal width ratios (Metrics \textbf{B4÷B1}, \textbf{B4÷B2}, \textbf{B4÷B3}) showed a significant difference between species.

For the metric B1 C. kalinini has an average of 5.49 $mm$ and a standard deviation of  0.20 $mm$. C. resplendens has an average of 5.68 $mm$ and a standard deviation of  0.21 $mm$. The difference between species is statiscally significant
For the metric B2 C. kalinini has an average of 8.24 $mm$ and a standard deviation of  0.27 $mm$. C. resplendens has an average of 8.56 $mm$ and a standard deviation of  0.35 $mm$. The difference between species is statiscally significant
The angle of the pronotum, as seen from its side (C1), has no significant difference between species.The angle of the pronotum, or as seen from the top (B5), has no significant difference between species.\subsubsection*{Claim 3}
\textit{Mesosternal process shiny, shorter and wider than in \textit{C. resplendens}, where the process is long and narrow and its apical half is greenish golden (Fig. 6--8).}

The first approach is to interpret the claim as a statement about the width-length ratio of the mesosternal process. 

The vertical length base width ratio, \textbf{D3÷D1},  is not statistically significant 

The vertical length down to the vertex of the dark stripe of the mesosternal process- horizontal length of the dark stripe, \textbf{D4÷D2}

, is not statistically significant.There is a significant difference in  the absolute vertical length values between the two species (Metric \textbf{D2}, Figure 23). There is no significant difference in their widths (Metrics \textbf{D1} and \textbf{D3}). There is a significant difference between species in the vertical distance between the tip of the mesosternal process and the lower point of the dark curve in its middle.

\subsubsection*{Claim 4}
\textit{Prosternal plate of \textit{C. kalinini sp.n.} is rounded triangular and flat, whereas in \textit{C. resplendens} it is square and has a clear dent.}

Although there is a difference between the absolute values of the foremost width of the prosternal process (Metric \textbf{E1}, Figure 29), there is no significant difference in how square the prosternal plate is for each species when the ratio of lengths is taken into account (Metric \textbf{E1÷E2}, ratio between \textbf{E1} and \textbf{E2}; Figure 51).

\subsubsection*{Other metrics}
Other metrics of interest, not directly related to any of Zubov et al's claims are the following:

For the metric D4 C. kalinini has an average of 0.73 $mm$ and a standard deviation of  0.12 $mm$. C. resplendens has an average of 0.86 $mm$ and a standard deviation of  0.14 $mm$. The difference between species is statiscally significant


For the metric E1 C. kalinini has an average of 0.50 $mm$ and a standard deviation of  0.06 $mm$. C. resplendens has an average of 0.58 $mm$ and a standard deviation of  0.09 $mm$. The difference between species is statiscally significant


For the metric F1 C. kalinini has an average of 1.40 $mm$ and a standard deviation of  0.04 $mm$. C. resplendens has an average of 1.46 $mm$ and a standard deviation of  0.09 $mm$. The difference between species is statiscally significant


For the metric F2 C. kalinini has an average of 1.40 $mm$ and a standard deviation of  0.07 $mm$. C. resplendens has an average of 1.48 $mm$ and a standard deviation of  0.09 $mm$. The difference between species is statiscally significant


For the metric F3 C. kalinini has an average of 1.45 $mm$ and a standard deviation of  0.09 $mm$. C. resplendens has an average of 1.54 $mm$ and a standard deviation of  0.09 $mm$. The difference between species is statiscally significant


For the metric F4 C. kalinini has an average of 1.94 $mm$ and a standard deviation of  0.18 $mm$. C. resplendens has an average of 2.18 $mm$ and a standard deviation of  0.24 $mm$. The difference between species is statiscally significant


\newpage

\subsection{A word of caution}

Even though there are multiple significant metrics, all of them are unfeasible to be used in the field given that most of these differences are of less than 1 mm in length.:

\newpage

