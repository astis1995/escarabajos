\section{Methodology}

Chrysina samples were retrieved from the following locations: \\ 
\textit{kalinini}: AM, nan \\ 
\textit{resplendens}: MV \\ 

Sex distribution is as follows:
\begin{itemize}
  \item \textit{C. kalinini}: 4 males, 0 females, 7 unknown
  \item \textit{C. resplendens}: 13 males, 10 females, 0 unknown
\end{itemize}

Using an estereoscope (Resolution 4.781 $\mu$m per pixel), its head, clipeum, mesosternal process, prosternal process, and ventral plates were measured. 

An OCR software was used to retrieve the measurements and to add contextual information about collection location, sex, genus, and species. 

Zubov et al.'s morphological differences were calculated using the metrics taken with the estereoscope. 

For the statistical analysis, the measurement dataset is filtered to include only entries that correspond to either \textit{resplendens} or \textit{kalinini} specimens. 

For each metric, \texttt{NA} values are dropped, and a Shapiro-Wilk test for normality is performed for both species.

If the p-value of the Shapiro-Wilk test is greater than 0.05, the dataset can be assumed to be normal.

If both datasets are normal, a Levene test is performed to check for homogeneity of variances. 

If the value for the Levene’s test is greater than 0.05, variances are deemed to be equal.

At this point, one of the following cases will occur:

\begin{itemize}
    \item If the datasets are normal with equal variances, a Student’s t-test is applied.
    \item If the datasets are normal with different variances, a Welch’s t-test is applied.
    \item If at least one group is not normal, a Mann-Whitney U test is applied.
\end{itemize}

For all three tests, if the p-value is less than 0.05, there is no significant difference.

\newpage
